\documentclass[10pt,a4paper]{moderncv}
\usepackage[utf8]{inputenc}
\moderncvstyle{casual}
\usepackage{etoolbox}
\patchcmd{\makefoot}% <cmd>
{0.8\textwidth}% <search>
{\textwidth}% <replace>
{}{}% <success><failure>
\moderncvcolor{blue}
\usepackage{pdfpages}
\usepackage{xpatch}
\setlength{\hintscolumnwidth}{70pt}
\usepackage[scale=0.9]{geometry}
\usepackage{bold-extra}
\firstname{Dr.~Julian}
\familyname{Geiger}
\title{Curriculum Vitae}
% \photo[70pt][0pt]{appl_photo_crop.png}
\photo[70pt][0pt]{psi_photo_crop.png}
\usepackage{lmodern}
\usepackage[T1]{fontenc}
\usepackage[utf8]{inputenc}
\usepackage{fontawesome5}
% \usepackage{academicons}
\nopagenumbers
\renewcommand*{\httplink}[2][]{%
  \ifthenelse{\equal{#1}{}}%
  {\href{https://#2}{#2}}%
  {\href{https://#2}{#1}}}

% Underline only
\newcommand\Colorhref[3][cyan]{\href{#2}{\small\underline{#3}}}

% Social icons
\social[github][github.com/GeigerJ2/]{GeigerJ2}
\social[googlescholar][scholar.google.com/citations?user=UxLDe2oAAAAJ]{Julian
  Geiger}
\social[orcid][orcid.org/0000-0003-0023-1960]{ORCID}
\social[linkedin][linkedin.com/in/julian-geiger-12b267198/]{julian-geiger}

\begin{document}
\xpatchcmd\cventry{,}{}{}{}
\xpatchcmd\cventry{.}{}{}{}
\makecvtitle

\vspace{-4em}
\section{General Information}
\cvitem{E-Mail}{\textnormal{julian.geiger@psi.ch}}
\cvitem{Phone}{\textnormal{(+41)\,775061926}}
\cvitem{Residence}{\textnormal{8458 Dorf, Z\"urich, Switzerland}}
\cvitem{Nationality}{\textnormal{German}} \cvitem{Birth}{\textnormal{Date: 30th
  Nov 1994. Location: 90471 Nuremberg}}

\section{Education}
%
\cvitem{02/24\,--\,01/26}{\textbf{Postdoctoral Researcher / Research Software
  Engineer} Paul-Scherrer Institute (PSI), Materials Software and Data
  Group of
  Dr.~Giovanni Pizzi, Switzerland}
%
\cvitem{02/23\,--\,05/23}{\textbf{Research internship} EPFL, Nicola Marzari research
  group, Switzerland}
% High-throughput study on the self-consistent evaluation of Hubbard U and V parameters of olivine and spinel Li-ion battery materials using Density Functional Perturbation Theory
%
\cvitem{10/19\,--\,10/23}{\textbf{Chemistry PhD} Institute of Chemical Research of Catalonia
  (ICIQ), N{\'u}ria L{\'o}pez research
  group, Spain}
% Ceria-based Single-Atom Catalysts: From simplified models towards real world complexity
%
\cvitem{06/18\,--\,12/18}{\textbf{Master Thesis} BASF, Ansgar Sch{\"a}fer research
  group (ROM/CQ), Germany, Grade: 1.0}
% Structure generation for the automated
% investigation of reaction networks using single-ended reaction path optimization algorithms
%
\cvitem{10/17\,--\,03/18}{\textbf{Erasmus semester} University of Cambridge, Michiel
  Sprik research group, UK}
% Density Functional Theory study on the band alignment of anatase-TiO$_2$ surfaces in contact with water
%
\cvitem{10/16\,--\,12/18}{\textbf{Chemistry M.Sc.} Friedrich-Alexander-University
  Erlangen-Nuremberg (FAU), Germany, Grade: 1.0}

%
\cvitem{04/16\,--\,07/16}{\textbf{Bachelor thesis} FAU, Andreas G{\"o}rling research
  group, Germany, Grade: 1.0}
% Density Functional Theory investigations on potential catalysts for CO$_2$ activation on the basis of Cu(I)--NHC complexes with \textsc{Turbomole}
%
\cvitem{10/13\,--\,10/16}{\textbf{Chemistry B.Sc.} FAU, Germany, Grade: 1.2}
%
\cvitem{08/05\,--\,07/13}{\textbf{A-levels} Adam-Kraft-Gymnasium
  Schwabach, Germany, Grade: 1.4}

\section{Work as a Research Software Engineer}

% aiida-core
\cvitem{\texttt{aiida-core}
    \Colorhref{https://github.com/aiidateam/aiida-core}{\faIcon{github}}}{Contributor to and maintainer of the AiiDA workflow manager (large-scale project with >200k lines of Python code)}
\cvitem{}{
  \begin{itemize}
    \renewcommand{\labelitemi}{$\triangleright$}
    \item Collaborative work in an intermediate-sized team ($\approx$\,15 currently active members) on open-source, version-controlled software (using \texttt{git} and GitHub)
    \item General feature additions, usability improvements, bug
          fixes, and maintenance
    \item Focus on AiiDA's Object-Relational Mapper (ORM) implementation and data storage modules
    \item Implementation of a feature to extract data from AiiDA's SQL database (PostgreSQL / SQLite)
          and machine-readable \Colorhref{https://github.com/aiidateam/disk-objectstore/}{disk-objectstore} into human-readable directory trees (\Colorhref{https://github.com/aiidateam/aiida-core/pull/6723}{9k lines contribution})
    \item Familiarity with continuous integration (CI) tools such as pre-commit and GitHub Actions
    \item Experience with the full code quality infrastructure in place in AiiDA core: testing (pytest), linting (ruff),
    and static type checking (mypy)
  \end{itemize}}

% aiida-workgraph
\cvitem{\texttt{aiida-
      workgraph}
      \Colorhref{https://github.com/aiidateam/aiida-workgraph}{\faIcon{github}}}{Contributor to the AiiDA WorkGraph
      package to simplify dynamic workflow construction with AiiDA
  \vspace{0.5em}
  \begin{itemize}
    \renewcommand{\labelitemi}{$\triangleright$}
    \item General API discussions, implementation of new features,
          documentation updates
    \item Integration of the CI pipeline used in AiiDA core into the AiiDA WorkGraph package
  \end{itemize}
}

% sirocco
\cvitem{\texttt{sirocco}
    \Colorhref{https://github.com/C2SM/Sirocco}{\faIcon{github}}}{AiiDA-based Python workflow tool for weather \& climate
  simulations}
\cvitem{}{
  \begin{itemize}
    \renewcommand{\labelitemi}{$\triangleright$}
    \item Declarative, YAML-based workflow specification
    \item \texttt{pydantic} models for parsing and validation of workflow files
    \item \texttt{hatch} for Python project management
  \end{itemize}
}

% renku
\cvitem{\texttt{renku integration}
    \Colorhref{https://github.com/aiidateam/renku2-aiida-integration}{\faIcon{github}}}{Integration of AiiDA and the
    \Colorhref{https://renkulab.io/}{renkulab.io} platform for the exploration of AiiDA archives hosted on Materials Cloud Archive (\Colorhref{https://archive.materialscloud.org/}{MCA}) using a containerized deployment
}

\vspace{1em}
% basf
\cvitem{\texttt{basf/\\precomplex} \Colorhref{https://github.com/basf/precomplex_generator}{\faIcon{github}}}{Python and \textsc{Fortran} code for obtaining suitable input
  structures for automated transition-state searches for reaction path optimization algorithms (code in daily use in an industrial setting since 2018)}

% misc
\cvitem{Honorable mentions}{
  \begin{itemize}
    \renewcommand{\labelitemi}{$\triangleright$}
    \item Administrator of the group's High Performance Computing (HPC) cluster (10 nodes) using YAML configuration files to
    configure user accounts, SLURM, and a BeeGFS shared file system
    \item Supervision of Master student's semester project on optimizations
          to AiiDA's ORM layer and database interactions
    \item Responsible person for the AiiDA
          \Colorhref{https://linkedin.com/company/aiidateam}{LinkedIn page}
          and the
          publication of a \Colorhref{https://aiida.net/posts}{blog post} series
    \item Organization of regular AiiDA coding days and brainstorm
          meetings for team organization and planning
    \item Experience with various Quantum Chemistry simulation codes: VASP, Quantum ESPRESSO, CP2K, and Turbomole on HPC systems
    \item English proficiency (C2 level) and basic Spanish communication skills
  \end{itemize}
}

% geeky stuff
\cvitem{Geeky stuff I like}{
  \begin{itemize}
    \renewcommand{\labelitemi}{$\triangleright$}
    \item Currently learning the \texttt{rust} programming language
    \item Currently learning \Colorhref{https://airflow.apache.org}{Apache Airflow} in a team effort to replace AiiDA's
    internal engine by an industrial, general-purpose workflow manager
    \item Author of the AiiDA plugin for the \texttt{fish} shell (\texttt{plugin-aiida} \Colorhref{https://github.com/GeigerJ2/plugin-aiida}{\faIcon{github}})
    \item AstroNvim as IDE (with plugins, of course)
    \item \texttt{fish} shell with \texttt{tmux} as terminal environment
    \item Keyboard-driven \texttt{qutebrowser}
  \end{itemize}
}

\section{Scholarships}

\cvitem{08/16\,--\,10/16}{\textbf{DAAD RISE} 6-week chemistry internship at the Texas-Tech-University (Lubbock, USA)}
\cvitem{2015\,--\,2018}{\textbf{Deutschlandstipendium} funded by the German
  government and the Leonhard Kurz Stiftung \& Co. KG}

\section{Conferences}

\cvitem{13\,--\,14/08/2024}{Hands-on \Colorhref{https://nanohub.org/resources/40427\#workshop}{workshop} and
\Colorhref{https://nanohub.org/resources/40533}{panel discussion} on FAIR Workflows in Materials Science (Purdue
University, USA)}
\cvitem{03\,--\,07/07/2023}{IUVSTA-ZCAM: Metal-oxide ultrathin films and nanostructures: Experiment meets theory (Zaragoza, Spain): Poster contribution}
\cvitem{22\,--\,25/08/2022}{Psi-k conference (Lausanne, Switzerland): Poster contribution}

\section{Tutorials}

%
\cvitem{13/08/2024}{Hands-on session on AiiDA as part of the FAIR Workflows workshop (\Colorhref{https://nanohub.org/resources/40504}{part 1} and \Colorhref{https://nanohub.org/resources/40505}{part 2})}
\cvitem{25/04/2024}{Tutorial on shell configuration \Colorhref{https://github.com/GeigerJ2/msd-gm-shell}{\faIcon{github}}}
\cvitem{29/06/2022}{Introduction to VS Code \Colorhref{https://www.youtube.com/watch?v=d5ELLGlftB0}{\faIcon{youtube}}}
\cvitem{21/04/2021}{Introduction to interactive visualization of scientific data with Plotly \Colorhref{https://www.youtube.com/watch?v=A2swENqThIY}{\faIcon{youtube}}}

\section{Scientific Publications}

\cvitem{2025}{Janssen, J., George, J., \textbf{Geiger, J.} et al. A Python
  workflow definition for computational materials design. \textit{arXiv
    preprint} \Colorhref{https://arxiv.org/abs/2505.20366}{arXiv:2505.20366}.}
%
\cvitem{2024}{Minotaki, M., \textbf{Geiger, J.} et al., A
  generalized model for estimating adsorption energies of single atoms on
  doped
  carbon materials \textit{J. Mater. Chem. A.} \textbf{18}, 2211260
  (2024).}
%
\cvitem{2023}{Yang, Q., Surin, I., \textbf{Geiger, J.} et al.
  Lattice-Stabilized Chromium Atoms on Ceria for N$_2$O Synthesis.
  \textit{ACS
    Catal.} \textbf{13}, 15977--15990 (2023).}
%
\cvitem{2023}{\textnormal{Surin, I., \textbf{Geiger, J.} et al.,
    Low-Valent Manganese Atoms Stabilized on Ceria for Nitrous
    Oxide Synthesis.
    \textit{Adv. Mater.} \textbf{35}, 2211260 (2023).}}
%
\cvitem{2022}{\textbf{Geiger, J.}, Sabadell-Rend{\'o}n, A., Daelman, N. et al.
Data-driven models for ground and excited states for Single Atoms on Ceria.
\textit{npj Comput. Mater.} \textbf{8}, 171 (2022).}
%
\cvitem{2022}{\textnormal{\textbf{Geiger, J.} \& L{\'o}pez, N. Coupling
    Metal and Support Redox Terms in Single-Atom Catalysts.
    \textit{J. Phys.
      Chem.
      C} \textbf{126}, 13698–13704 (2022).}}
%
\cvitem{2022}{\textnormal{\textbf{Geiger, J.}, Settels, V., Deglmann, P. et al.
    Automated input structure generation for single-ended reaction
    path
    optimizations. \textit{J. Comput. Chem.} \textbf{43}, 1662-1674
    (2022).}}
%
\cvitem{2022}{\textnormal{Wan, W., \textbf{Geiger, J.} et al. Highly
    Stable and Reactive Platinum Single Atoms on Oxygen
    Plasma-Functionalized
    CeO$_2$ Surfaces: Nanostructuring and Peroxo Effects.
    \textit{Angew. Chem.
      Int.
      Ed.} \textbf{61}, e202112640 (2022).}}
%
\cvitem{2020}{\textbf{Geiger, J.}, Sprik, M., May, M. M. et al. Band positions
  of anatase (001) and (101) surfaces in contact with water from density
  functional theory. \textit{J. Chem. Phys.} \textbf{152}, 194706
  (2020).}

\end{document}
